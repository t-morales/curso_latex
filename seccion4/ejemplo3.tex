 \documentclass[a4paper,12pt]{article}
 \usepackage[latin1]{inputenc}
 \usepackage[T1]{fontenc}
 \usepackage[spanish]{babel}
 \usepackage[pdftex]{graphicx}
 \usepackage{wrapfig}


\begin{document}
\begin{wrapfigure}{r}{4.0cm}
\includegraphics[width=4.0cm]{coche}
\caption[TextoLeyendaIndice]{Leyenda} \label{etiqueta}
\end{wrapfigure}
Al incluir una figura con el comando wrapfigure se deben tener
algunas cosas en cuenta: En la definici�n \{r\} significa que el
recuadro se introducir� a la derecha del texto, tambi�n se puede
utilizar \{l\} para que sea a la izquierda. El entorno se debe
iniciar entre p�rrafos, es decir, es problem�tico escribir un
entorno wrapfigure en medio de un p�rrafo. La figura ser�
introducida justo al lado del p�rrafo siguiente de la definici�n
del entorno.

\begin{wrapfigure}{l}{2.0cm}
\includegraphics[width=2.0cm]{coche}
\end{wrapfigure}
Al incluir una figura con el comando wrapfigure se deben tener
algunas cosas en cuenta: En la definici�n \{r\} significa que el
recuadro se introducir� a la derecha del texto, tambi�n se puede
utilizar \{l\} para que sea a la izquierda. El entorno se debe
iniciar entre p�rrafos, es decir, es problem�tico escribir un
entorno wrapfigure en medio de un p�rrafo. La figura ser�
introducida justo al lado del p�rrafo siguiente de la definici�n
del entorno.
 \end{document}
